\beginrecipie{Chana Masala}{10}{60}{6}
\begin{ingredient}
    \begin{main}
        \item 5 tins canned chickpeas
        \item 2 tins chopped tomatoes
        \item 3 tablespoons ginger
        \item 10 cloves garlic
        \item 3 green chillies
        \item 3 bay leaves
        \item 12 green cardamom pods
        \item 9 cloves
        \item 2 cinnamon sticks
        \item \tablespoons{1} black peppercorns
        \item \tablespoons{3} cumin seeds
        \item 3 large red onions
        \item \tablespoons{3} tomato paste
        \item \tablespoons{1 1/2} coriander powder
        \item \teaspoons{2} cumin powder
        \item \teaspoons{2} paprika
        \item \teaspoons{1} turmeric
        \item \teaspoons{1} garam masala
        \item salt
    \end{main}
\end{ingredient}
\begin{recipe}
    \step{Heat a pot on medium, add some oil and the whole spices- bay leaf, green cardamom, cloves, cinnamon, peppercorns and cumin seeds.}
    \step{Add the ginger, garlic and green chillies and allow to fry for a few minutes before adding the onions.}
    \step{Once the onions have softened add the chopped tomatoes and cook for 5 minutes.}
    \step{Add the ground spices --- coriander powder, cumin powder, paprika, turmeric and garam masala. Cook the spices for 30 seconds.}
    \step{Stir in the boiled chickpeas and mix. Add 5 cups of water and stir.}
    \step{Simmer for 10--15 minutes for the flavors to mix-in together. Serve with basmati rice or saag aloo.}
\end{recipe}
%https://www.cookwithmanali.com/chana-masala/
