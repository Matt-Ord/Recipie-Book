
%------------------------------------------
% information doc
% \PrepTime{50}
% \CookingTime{10}
% \CookingTempe{180}
% \TypeCooking{Plaque}
% \NbPerson{4}
% \Image{0 100 400 200}{images/florentin} %style 1
%\Image{0 100 800 900}{images/lasagnes} %style 2
%------------------------------------------
\beginrecipie{Florentins au chocolat}{50}{10}{4}

\begin{ingredient}
	%\vspace{0.5cm}
	\begin{main}
		\item 4 cà.S de sucre en poudre
		\item 1 cà.S de crème liquide
		\item 1 cà.S de miel
		\item 1 grosse noix de beurre
		\item 35 gr d’amandes effilées
		\item 50 gr de chocolat au lait ou chocolat blanc
	\end{main}
	\begin{subingredient}{Test subingredient}
		\item 1 cà.c de test1
		\item 1 à 2 cà.S de test2
		\item 3 gouttes de test3
		\item 8 morceaux de test4.
	\end{subingredient}
\end{ingredient} %no space with \begin{recipe}
\begin{recipe}
	\step{Préchauffez votre four à 180°C (th.6).}
	\step{Dans une casserole, faites bouillir le sucre en poudre avec la crème liquide, le beurre et le miel.}
	\step{Une fois que le sucre prend une jolie coloration brune, versez les amandes effilées dans la casserole, et remuez bien pour napper l’intégralité des amandes.}
	\step{Pour la cuisson au four vous avez 2 possibilités: Soit vous versez la « pâte » dans le fond de moules en silicone, type moules à muffins ou moules à tartelettes, soit vous étalez bien la « pâte », et rapidement car le caramel durcit vite, sur la plaque de votre four recouverte d’une feuille de papier sulfurisé, et après la cuisson vous découperez des cercles à l’aide d’un emporte-pièces rond.}
	\step{Dans tous les cas, mettez la « pâte » au four pendant 3 à 5 minutes. A la sortie du four, soit vous découpez tout de suite des ronds à l’aide de l’emporte-pièces, soit vous laissez refroidir les florentins avant de les démouler de vos moules à muffins.}
	\step{Pendant que les florentins refroidissent, faites fondre le chocolat au lait ou blanc soit au bain-marie, soit au micro-ondes à faible puissance, soit dans une petite casserole à feu doux.}
	\step{Trempez ensuite la moitié des florentins dans le chocolat fondu et mettez-les au réfrigérateur pendant une bonne vingtaine de minutes pour que le chocolat prenne bien.}
\end{recipe}

% \begin{notes}

% \end{notes}