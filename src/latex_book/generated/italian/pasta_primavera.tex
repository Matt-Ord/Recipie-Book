\beginrecipie{Pasta Primavera}{30}{30}{4}
\begin{ingredient}
    \begin{main}
        \item \grams{180} English Peas
        \item \grams{180} Green Beans
        \item \grams{200} Asparagus
        \item \grams{160} Tender Stem Broccoli
        \item Butter
        \item \grams{100} Pine Nuts
        \item 2 cloves Garlic
        \item Penne Pasta
        \item \ml{170} Crème Fraîche
        \item 1 Lemon
        \item \cups{0.5} fresh Basil
        \item \cups{0.5} fresh Parsley
        \item Parmesan Cheese
    \end{main}
\end{ingredient}
\begin{recipe}
    \step{Bring a large pot of water to boil and add the green beans. After a minute or two add the pasta and cook for a minute before adding the Asparagus and Broccoli.}
    \step{Heat the butter in a pan and lightly toast the Pine Nuts along with the Garlic until lightly browned. Discard cloves and remove from heat.}
    \step{Once the pasta is almost done add the peas and allow the water to return to boil. If using fresh peas these should be added earlier}
    \step{Once almost cooked transfer the pasta and vegetables to the pan along with a little of the cooking liquid.}
    \step{Add the Crème Fraîche, basil, parsley, lemon juice and 2 teaspoons of the lemon zest.}
    \step{Stir constantly over high heat, adding starchy water as necessary to achieve a creamy consistency.}
\end{recipe}