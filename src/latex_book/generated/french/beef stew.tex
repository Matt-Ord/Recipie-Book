\beginrecipie{Beef Stew}{20}{120}{4}
\begin{ingredient}
    \begin{main}
        \item \grams{750} beef steak
        \item \tablespoons{2} flour
        \item 2 cloves garlic
        \item \grams{175} onions
        \item \grams{150} celery
        \item \grams{150} carrots
        \item 2 leeks
        \item \grams{200} swede
        \item \ml{150} red wine
        \item \ml{500} beef stock
        \item 2 bay leaves
        \item \tablespoons{3} fresh thyme
        \item \tablespoons{3} fresh parsley
        \item Worcestershire sauce
    \end{main}
    \begin{subingredient}{dumplings}
        \item \grams{125} plain flour
        \item \teaspoons{1} baking powder
        \item \grams{60} suet
    \end{subingredient}
\end{ingredient}
\begin{recipe}
    \step{Preheat the oven to 180°C.}
    \step{For the beef stew, heat the oil and butter in an ovenproof casserole and fry the beef until browned on all sides.}
    \step{Sprinkle over the flour and cook for a further 2--3 minutes. Add the garlic and all the vegetables and fry for 1--2 minutes.}
    \step{Stir in the wine, stock and herbs, then add the Worcestershire sauce and balsamic vinegar, to taste. Season with salt and freshly ground black pepper.}
    \step{Cover with a lid, transfer to the oven and cook for about two hours, or until the meat is tender.}
    \step{For the dumplings, sift the flour, baking powder and salt into a bowl. Add the suet and enough water to form a thick dough.}
    \step{With floured hands, roll spoonfuls of the dough into small balls.}
    \step{After two hours, remove the lid from the stew and place the balls on top of the stew. Cover, return to the oven and cook for a further 20 minutes, or until the dumplings have puffed up and are tender. (If you prefer your dumplings with a golden top, leave the lid off when returning to the oven.)}
    \step{To serve, place a spoonful of mashed potato onto each of four serving plates and top with the stew and dumplings. Sprinkle with chopped parsley.}
\end{recipe}